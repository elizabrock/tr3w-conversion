

\chapter{Rails Environments and Configuration}
\begin{quote}
[Rails] gained a lot of its focus and appeal because I didn't try to
please people who didn't share my problems. Differentiating between
production and development was a very real problem for me, so I solved
it the best way I knew how.

--David Heinemeier Hansson
\end{quote}
Rails applications are preconfigured with three standard modes of
operation: development, test, and production. These modes are basically
execution environments and have a collection of associated settings
that determine things such as which database to connect to, and whether
or not the classes of your application are reloaded with each request.
It is also simple to create your own custom environments if necessary.

The current environment is always specified in the environment
variable \verb|RAILS_ENV|, which names the desired mode of operation
and corresponds to an environment definition file in the
\verb|config/environments| folder. Since the environment settings
govern some of the most fundamental aspects of Rails, such as
classloading, in order to really understand the Rails way you should
really understand its environment settings.

In this chapter, we get familiar with how Rails starts up and handles
requests, by examining scripts such as \verb|boot.rb| and
\verb|application.rb| and the settings that make up the three standard
environment settings (modes). We also cover some of the basics of
defining your own environments, and why you might choose to do so.

\section{Startup and Application Settings}
Whenever you start a process to handle requests with Rails (such as
with \verb|rails |\verb|server), one of the first things that happens
is that config/|\verb|boot|\verb|.rb is loaded.|

Processes require different components depending on what they need.
Some processes need the whole Rails environment, so they require
\verb|config/environment.rb|\verb|, you can see an example of that at
the top of files such as |\verb|spec|\verb|/|\verb|spec|\verb|_helper.rb.|

\subsection{application.rb}
The file \verb|environment.rb| used to be where many of your
application settings lived. In Rails 3, the settings move to a file
called \verb|application.rb|, and it's the only file required at the
top of \verb|environment.rb|.

Let's go step by step through the settings provided in the default
\verb|application.rb| file that you'll find in a newly created Rails
application. By the way, as you're reading through the following
sections, make a mental note to yourself that changes to these files
require a server restart to take effect.

\subsection{boot.rb}
The next lines of \verb|application.rb| are where the wheels really
start turning, once \verb|config/boot.rb| is loaded:

\begin{code}
\begin{Verbatim}
require File.expand_path('../boot', __FILE__)
\end{Verbatim}
\end{code}
Note that the boot script is generated as part of your Rails
application, but you won't usually need to edit it. It takes care of a
two essential steps in starting up the environment:

\begin{itemize}
\item Require gems required by Rails and your application according to its
Bundler configuration

\item Require the Rails component frameworks.

\end{itemize}
Usually, the following line (default) will suffice:

\begin{code}
\begin{Verbatim}
require 'rails/all'
\end{Verbatim}
\end{code}
However, a new feature of Rails 3 is the ability to easily cherry-pick
only the components needed by your application.

\begin{code}
\begin{Verbatim}
# To pick the frameworks you want, remove 'require "rails/all"'
# and list the framework railties that you want:
#
# require "active_model/railtie"
# require "active_record/railtie"
# require "action_controller/railtie"
# require "action_view/railtie"
# require "action_mailer/railtie"
# require "active_resource/railtie"
\end{Verbatim}
\end{code}
